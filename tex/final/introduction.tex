\section{Introduction}
% \addcontentsline{toc}{chapter}{1. Introduction}
Il est bien connu que la s\'erie harmonique
\[
	\sum_{n=1}^{\infty} \frac{1}{n} = 1 + \frac{1}{2} + \frac{1}{3} +
	\frac{1}{4} + \cdots
\]
qui consiste en la somme infinie de la r\'eciproque des entiers naturels non
nuls est divergente. Cependant, en 1914, Kempner \cite{kempner} a ajout\'e une
restriction sur les termes de cette somme et s'est demand\'e ce qu'il se passe
si on \'elimine les \'el\'ements dont le d\'enominateur contient le chiffre 9.
C'est-\`a-dire la somme $K = 1 + 1/2 + \cdots + 1/8 + 1/10 + \cdots$. La r\'eponse
est que la s\'erie converge, ce que Kempner a prouv\'e de mani\`ere simple et
\'el\'egante. Mais il n'a pas donn\'e d'indication sur la valeur vers laquelle
cette s\'erie converge, \`a part montrer que cette valeur est inf\'erieure \`a
90.

% Just after two years, Irwin published his article \cite{irwin} where he cleverly
% regroups the sum and find new upper and lower bound. Originally, he found
% correct integer for the sum by hand, and I can find up to 6 decimal points using
% computers.
Deux ans plus tard, Irwin a publi\'e son article \cite{irwin} dans lequel il
regroupe les termes de la somme et trouve de nouvelles bornes sup\'erieures et
inf\'erieures. \`A l'origine, il a trouv\'e correct pour la somme \`a la main,
et avec sa m\'ethode nous pouvions obtenir jusqu'\`a 6 chiffres apr\`es la
virgule \`a l'aide d'un ordinateur.

% After, more than 60 years later, Baillie came up for a new and very important
% key idea to work with $1/x^{j}$ instead of $1/x$ and rewrite $x$ as $10\lfloor
% x/10 \rfloor + (x\Mod10)$. Then, by using binomial theorem, he created a
% recurrence relation between the sum of reciprocals and the sums of powers of
% reciprocals. His clever method will be the key for the rest of the methods. This
% method can be easily generalized to other digits. Baillie use his method to
% calculate up to 20 decimal points for all missing digits.
Soixante ans plus tard, Baillie \cite{baillie} a propos\'e une nouvelle id\'ee
cl\'e: travailler avec $1/x^{j}$ en lieu de $1/x$ et r\'e\'ecrire l'entier $x$
comme $10\lfloor x/10 \rfloor + (x\Mod10)$. Puis, en utilisant la formule
binomiale, il cr\'ee une relation r\'ecurrence entre la somme des inverses et la
somme de puissances des inverses. Sa m\'ethode astucieuse sera la cl\'e pour le
reste des m\'ethodes. Cette m\'ethode peut \^{e}tre g\'en\'eralis\'ee pour les
autres chiffres. Baillie utilise sa m\'ethode pour calculer jusqu'\`a 20 chiffres
apr\`es la virgule pour le probl\`eme original et des variantes.

% Then, Fischer turn back to the original question and leave the generalizations
% behind. Fischer worked with the function $1/(m+x)$ and proposed two different
% approach for calculation of Kempner Sum. In first method, he uses advanced tools
% in analysis such as Digamma Function and Riemann Zeta Function and found another
% recursive formula to calculate the values by a very simple recursion formula. In
% his second method, he uses Chebyshev Polynomials of the first kind to squeez the
% sum into an upper and lower limit for the sum. His methods was extremely fast
% but can not be easily generalized into other cases than Kempner.
Puis, Fischer \cite{fischer} retourne à la question originale et laisse les
généralisations de côté. Fischer a travaillé avec la fonction $1/(m+x)$. En
définissant un op\'erateur lin\'eaire, il a proposé deux approches différentes pour
le calcul de la somme de Kempner. Dans la première méthode, il utilise des
outils avancés en analyse tels que la fonction digamma et la fonction zêta de
Riemann et a trouvé une autre formule pour calculer les valeurs par une
récurrence très simple. Dans sa deuxième méthode, il utilise les polynômes de
Chebyshev afin d'encadrer la somme entre une borne supérieure et une borne
inférieure. Ces méthodes sont extrêmement rapides mais ne peuvent pas être
généralisées facilement à d'autres cas que celui de Kempner.

% After Fischer's incredible work, Baillie works with Schmelzer to find a new
% algorithm to generalize the constraints which results in a faster algorithm.
% They mainly use Matrix Theory, Linear Algebra and a little bit Graph Theory in
% their article. They also mentioned the asymptotic behaviour of the digit
% restricted sequences, which proposed that no matter what is the restriction they
% put, the final result is depends on the period of the motif.
Apr\`es le travail de Fisher, Baillie travaille avec Schmelzer \cite{schmelzer}
pour trouver un nouvel algorithme qui g\'en\'eralise les contraintes possibles.
Les motifs permis par Schmelzer sont très variés, par exemple au lieu de juste
un chiffre, cette méthode de Schmelzer \& Baillie nous permet de restreindre
tous les entiers qui contient le motif “314”. Ils utilisent principalement
de l'alg\`ebre lin\'eaire matricielle, et la th\'eorie de graphes.
Ils mentionnent \'egalement le comportement asymptotique de la suite
restreinte de chiffres. Ils ont propos\'e que, le r\'esultat final ne
d\'epend que de la periode du motif de restriction.

% And finally, Jean-François Burnol came to the scene and put a state of the art
% algorithm for today's standarts. He generalized Fischer's method with the usage
% of measure theory, with his method, we have reached 100 000 digits accuracy. The
% measure he defined, comes from the probabilistic intuition, but we can also see
% that it is a solution to the Hausdorff moment problem for $\alpha_{n}$'s Fischer
% used. Monsieur Burnol also used moments extensively in his algorithm just like
% Fischer. Also the measure defined here also defines a linear functional on
% continuous functions on the interval $[0, 1]$, which is the same functional
% Fischer used. So we can say that idea of Monsieur Burnol is further abstracted
% and generalized version of the Fischer's method. He also mentioned this link
% between his work and Fischer's at the end of the article.
Enfin, Jean-François Burnol \cite{burnol} a proposé une formule et un algorithme
très général et très efficace, à la pointe des connaissances actuelles. Il a
généralisé la méthode de Fischer avec l’usage de la théorie de la mesure. La
mesure qu’il a défini vient de l’intuition probabiliste, mais on peut aussi voir
que c’est une solution au problème des moments de Hausdorff pour les
$\alpha_{n}$ utilisés par Fischer. Avec sa méthode, nous avons atteint une
précision de 100 000 chiffres. 

% Monsieur Burnol a aussi utilisé les moments de manière extensive dans son
% algorithme tout comme Fischer. De plus, la mesure définie ici définit également
% un fonctionnel linéaire sur les fonctions continues sur l’intervalle $[0, 1]$,
% qui est le même fonctionnel utilisé par Fischer. On peut donc dire que l’idée de
% Monsieur Burnol est une version plus abstraite et généralisée de la méthode de
% Fischer. Il a également mentionné ce lien entre son travail et celui de Fischer
% à la fin de l’article.


