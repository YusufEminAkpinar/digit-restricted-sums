\section{R\'esultats}
% \addcontentsline{toc}{chapter}{R\'esultats pr\'eliminaires}
\subsection{Somme de Kempner}
% \addcontentsline{toc}{section}{Somme de Kempner}
En 1914, Kempner a montr\'e le th\'eor\`eme suivant.\cite{kempner}
\begin{thm}
	La somme des inverses des entiers sans les termes qui contiennent le chiffre 9 dans
	leur d\'enominateur, appel\'ee somme de Kempner, converge.
\end{thm}
\begin{proof}
	Tout d'abord, nous partitionnons le somme selon que le nombre de chiffres
	dans le d\'enominateur et les nommant $S_{i}$. En d'autres termes, $S_{i}$
	sont les termes de somme de Kempner qui ont exactement $i$ chiffres dans
	leur d\'enominateur et d\'efinıssons $a_{i}$ comme la somme des termes de
	$S_{i}$.  Par exemple, les deux premiers termes sont,
	\begin{equation*}
		\begin{split}
			a_1 &= \sum_{s\in S_{1}} s = \frac{1}{1} + \frac{1}{2}
			+\frac{1}{3} + \cdots +\frac{1}{8} \\ a_2 &= \sum_{s\in S_{2}}
			s = \frac{1}{10} + \frac{1}{11} + \cdots + \frac{1}{18}
			+\frac{1}{20} + \cdots + \frac{1}{88}
		\end{split}
	\end{equation*}

	La premi\`ere observation que nous devons faire est que pour le premier
	chiffre, il y a 8 termes, et pour chacun des autres chiffres, il y a 9
	termes possibles. Nous avons donc 8 termes dans $a_{1}$ et $8\cdot9$ termes
	dans $a_{2}$. Plus g\'en\'eralement, il y a $8\cdot 9^{n-1}$ termes dans
	$a_{n}$. La deuxi\`eme observation est \`a propos des bornes des $a_{n}$,
	nous pouvons voir que tous ces termes sont plus petits que leur premier
	terme $\frac{1}{10^{n-1}}$. Alors, nous avons obtenu,
	\begin{equation*}
		\begin{split}
			\forall s\in S_{1}, s \le 1 &\Rightarrow a_{1} \le 1 \cdot 8 = 8\\
			\forall s\in S_{2}, s \le 1/10 &\Rightarrow a_{2} \le \frac{1}{10}
			\cdot 8 \cdot 9 = 8\cdot \frac{9}{10} \\
							  &  \vdots \\
			\forall s\in S_{n}, s \le 1/10^{n-1} &\Rightarrow a_{n} \le
			\frac{1}{10^{n-1}} \cdot 8 \cdot 9^{n-1} = 8\cdot \left(
			\frac{9}{10}\right)^{n-1}
		\end{split}
	\end{equation*}
	Avec une somme geom\'etrique simple, nous avons $K\le8\cdot10=80$. Alors on
	sait que la somme de Kempner converge et plus est petite que 80.
\end{proof}

Cela semble un peu contre-intuitif, mais c'est juste \`a cause de notre perception.
\`A premi\`ere vue, nous pensons surtout aux entiers avec un ou deux chiffres,
par lesquels il semble que juste une petite partie des termes sont \'ecart\'es.
Mais si nous \'elargissons notre vision \`a des nombres entiers plus grands,
disons avec 51 chiffres, seulement $\frac{8}{9} \cdot \left(9/10\right)^{50}
\approx 0,004 = 0,04\%$ n'est pas \'ecart\'e. Donc apr\`es un certain
temps, presque tous les termes sont \'ecart\'es.
