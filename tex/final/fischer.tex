\subsection{Plus rapide: Fischer}
% \addcontentsline{toc}{section}{Plus Rapide: Fischer}
En 1993, Hans-Jürgen Fischer \cite{fischer} a également travaillé sur les séries
de Kempner, mais pas sur des généralisations comme celles d'Irwin. Il a
découvert une méthode incroyablement rapide pour calculer de nombreux chiffres.
Nous présentons maintenant son idée.
\subsubsection{Équation de la fonction}
Soit $M$, l'ensemble des entiers dont les repr\'esentations d\'ecimales ne
contiennent pas le chiffre 9. Le nombre dans $M$, est soit un nombre à un
chiffre, qui est l'un de 1, 2, $\ldots$, 8 dans notre cas, soit il est formé en
ajoutant l'un de ces chiffres à un nombre dans M avec moins de chiffres. Ainsi,
nous avons :
% Let $M$ denote the set of integers whose decicmal representation does not contain
% the digit 9. Number in $M$, is either a single digit number, which is one of
% 1, 2, $\ldots$, 8 in our case, or it is formed by appending one of these
% digits to a number in M with fewer digits. Thus we have:
\begin{equation}
	M = \{1,2,3,4,5,6,7,8\} \cup \bigcup_{z=0}^8\{10m+z : m\in M\}
	\label{eq:setM}
\end{equation}
% For the precise calculation of the series \( \sum_{m \in M} \frac{1}{m} \), we
% consider the function
Pour le calcul pr\'ecis de la s\'erie \( \sum_{m \in M} \frac{1}{m} \), on
consid\`ere la fonction:
\[ 
	s(x) = \sum_{m\in M} \frac{1}{m+x}, x\ge 0, 
\]
qui, en raison de la convergence uniforme de la s\'erie pour $x>0$, est bien
d\'efinie et continue. De l'equation (\ref{eq:setM}), il r\'esulte que
% which due to the uniform convergence of the series for $x>0$, is well defined
% and continuous. From equation (\ref{eq:setM}), it follows that
\begin{equation*}
	\begin{split}
		s(x) &= \frac{1}{1+x} + \cdots + \frac{1}{8+x} + \sum_{z=0}^8
		\sum_{m\in M} \frac{1}{10m+z+x} \\
			 &= \frac{1}{1+x} + \cdots + \frac{1}{8+x} + \frac{1}{10}
			 \sum_{z=0}^8\sum_{m\in M} \frac{1}{m+\frac{z+x}{10}} \\
			 &= \frac{1}{1+x} + \cdots + \frac{1}{8+x} +
			 \frac{1}{10}\sum_{z=0}^8 s(\frac{z+x}{10}).
	\end{split}
\end{equation*}
% Thus, final equation is,
Alors, l'\'equation finale est,
\begin{equation}
	s(x) - \frac{1}{10}\sum_{z=0}^8 s(\frac{z+x}{10}) = \frac{1}{1+x} + \cdots +
	\frac{1}{8+x} = r(x).
	\label{eq:sx}
\end{equation}
On definit $r(x)$ comme le c\^{o}t\'e droit de l'\'equation (\ref{eq:sx}).
Nous pouvons continuer avec l'op\'erateur sur l'espace $C[0,1]$ d\'efini par:
% Introducing the notation $r(x)$ as an abbreviation for the right-hand side of
% equation (\ref{eq:sx}), we can proceed with the operator on the space $C[0, 1]$
% defined by 
\begin{equation}
	(Af)(x) = \frac{1}{10}\sum_{z=0}^{8} f(\frac{z+x}{10}).
	\label{eq:defA}
\end{equation}
% This operator allowing equation (\ref{eq:sx}) to be rewritten in the form
Cet op\'erateur permet de r\'e\'ecrire l'\'equation (\ref{eq:sx}) sous la forme
\begin{equation}
	s -As = r
	\label{eq:shortEq}
\end{equation}
\subsubsection{Quelques propri\'et\'es de l'\'equation de l'op\'erateur}
% \addcontentsline{toc}{subsection}{Quelques Propri\'et\'es de l'\'Equation de l'Op\'erateur}
% \noindent The follwoing properties of the operator $A$ are easy to verify:
\noindent Les propri\'et\'es suivantes de l'op\'erateur $A$ sont facile \`a
v\'erifier.
\begin{enumerate}
	\item $A$ est lin\'eaire.
	\item $\norm{A} = \frac{9}{10} < 1$.
	\item Si $f$ est non-n\'egatif sur $[0,1]$, alors c'est vrai aussi pour
		$Af$.
	\item Si $p_{n}$ est un polyn\^{o}me de degr\'e $n$, alors $Ap_{n}$ est aussi
		un polyn\^{o}me de degr\'e $n$.
\end{enumerate}
Les propri\'et\'es 1, 3 et 4 peuvent etre facilement v\'erifi\'ees en se
r\'ef\'erant \`a la d\'efinition (\ref{eq:defA}). Pour propri\'et\'e 2, on a,
% Properties 1, 3 and 4 are can be easily seen by just looking to (\ref{eq:defA}).
% For 2 we have,
\[
	\norm{A}_{op} = \sup \left\{\norm{Af}_{\infty} : \norm{f}_{\infty} =
	1\right\} = \frac{1}{10} \sum_{z=0}^{8}1 = \frac{9}{10} < 1.
\]
% Thus, operator $I-A$ is invertible, where $I$ denotes the identity operator.
Ainsi, l'op\'erateur $I-A$ est inversible, o\`u $I$ d\'esigne l'op\'erateur
identit\'e.
\begin{equation}
	(I-A)^{-1} = I + A + A^{2} + \ldots 
	\label{eq:IminAinv}
\end{equation}
et
\[ ||(I-A)^{-1}|| \le (1-||A||)^{-1} = 10 \]
% Thus the equation,
Alors l'\'equation,
\[ (I-A)f = g \]
% is uniquely solvable in $C[0, 1]$ for any $g\in C[0, 1]$, and our desired
% function $s(x)$ is uniquely determined by (\ref{eq:sx}) or (\ref{eq:shortEq}).
% Let $l$ be the function defined by
admet une solution unique dans $C[0,1]$ pour tout $g\in C[0,1]$, et notre
fonction d\'esir\'ee $s(x)$ est uniquement d\'etermin\'ee par (\ref{eq:sx}) (ou
(\ref{eq:shortEq})). Soit $l$ la fonction d\'efinie par
\[ l(g) = (I-A)^{-1}g(0) \qquad \text{ pour } g\in C[0,1], \]
% the defined functional, is obviously linear. From property 3 and equation
% (\ref{eq:IminAinv}), it follow that $g_{1}(x) \le g_{2}(x)$ for all $x \in
% [0,1]$, then $l(g_{1}) < l(g_{2})$ also holds. Moreover,
qui est \'evidemment lin\'eaire. De la propri\'et\'e 3 et l'\'equation
(\ref{eq:IminAinv}), il r\'esulte que $g_{1}(x) \le g_{2}(x)$ pour tout $x \in
[0, 1]$, alors $l(g_{1}) \le l(g_{2})$ aussi vrai. De plus,
\begin{equation}
	\norm{l} \le \norm{(I-A)^{-1}} \le 10 
	\label{eq:norml}
\end{equation}
en raison de
\[
	\sum_{m\in M} \frac{1}{m} = s(0) = (I-A)^{-1} r(0) = l(r),
\]
Notre t\^{a}che se r\'eduit donc \`a calculer $l(r)$. La base pour cela, puisque
propri\'et\'e 4, le $l(p)$ pour un polyn\^{o}me $p$ est facile \`a
calculer, et il est bien connu que les fonctions dans $C[0,1]$ peuvent \^{e}tre
approch\'ees arbitrairement pr\`es par des polyn\^{o}mes.
% our task thus boils down to calculating $l(r)$. The basis for this, due to
% propert 4, the functional $l(p)$ for a polynomial $p$ is easy to compute, and it
% is well-known that functions in $C[0,1]$ can be approximated arbitrarily closely
% by polynomials.
\subsubsection{Deux R\'ecurrences}
% \addcontentsline{toc}{subsection}{Deux R\'ecursions}
Pour un polyn\^{o}me donn\'e $p$, le polyn\^{o}me $q = (I-A)^{-1}p$,
d\'etermin\'e de mani\`ere unique, peut \^{e}tre trouv\'e facilement en
substituant dans l'\'equation $q-Aq = p$ et en comparant les coefficients. Mais,
nous seulement int\'eress\'es par $l(p)$, et $p(x)$ peut \^{e}tre \'ecrit comme
\( a_0 + a_1x + \dots + a_nx^n \) ou de façon \'equivalente comme \( b_0 + b_1(1
- x) + \dots + b_n(1 - x)^n \), il suffit de d\'eriver des relations
r\'ecursives pour les suites:
\[
	\alpha_{n} = l(x^{n}) \qquad \& \qquad \beta_{n} = l((1-x)^{n})
\]
Par d\'efinition, on a \( l((I - A)g) = g(0) \) pour tout $g\in C[0,1]$. Pour un
$t\in \N$ donn\'e, Soit \( g(x) = (e^{t/10} - 1)e^{tx} \), qui nous donne:
\begin{equation*}
	\begin{split}
		Ag(x) &= \frac{1}{10} (e^{t/10} - 1) \left( e^{tx/10} + \ldots +
		e^{t(x+8)/10} \right) \\
		&= \frac{1}{10} (e^{t/10} - 1) e^{tx/10} \left( 1 + e^{t/10} + \ldots +
		e^{8t/10} \right) \\
		&= \frac{1}{10} (e^{t/10} - 1) e^{tx/10}
		\frac{e^{9t/10}-1}{e^{t/10}-1} = \frac{1}{10}(e^{9t/10}-1)e^{tx/10} 
	\end{split}
\end{equation*}
et \`a partir de l\`a,
\[
	l \left( (e^{t/10}-1)e^{tx} - \frac{1}{10}(e^{9t/10}-1)e^{tx/10} \right) =
	e^{t/10}-1
\]
D\'eveloppans les deux c\^{o}t\'es en puissances de $t$:
\begin{align*}
	&l \left( (e^{t/10}-1)e^{tx} - \frac{1}{10}(e^{9t/10}-1)e^{tx/10} \right) =
	(e^{t/10} - 1)l( e^{tx})- \frac{1}{10}(e^{9t/10}-1)l(e^{tx/10}) \\
	&= \left( \sum_{n=1}^{\infty}\frac{t^{n}}{n!10^{n}}\right)
	\sum_{n=0}^{\infty} \frac{t^{n}}{n!}\alpha_{n} - \left(
	\sum_{n=1}^{\infty}\frac{9^{n}t^{n}}{n!10^{n}} \right) \sum_{n=0}^{\infty}
	\frac{t^{n}}{n! 10^{n+1}} \alpha_{n}\\
	&= \sum_{n=0}^{\infty}t^{n}\sum_{k=1}^{n}
	\frac{\alpha_{n-k}}{(n-k)!k!10^{k}} - \sum_{n=0}^{\infty} t^{n}
	\sum_{k=1}^{n} \frac{9^{k}}{k!} \frac{\alpha_{n-k}}{(n-k)!10^{n+1}}\\
	&= \sum_{n=0}^{\infty} t^{n} \sum_{k=1}^{n} \frac{\alpha_{n-k}}{(n-k)! k!}
	\left( \frac{1}{10^{k}} - \frac{9^{k}}{10^{n+1}} \right)
\end{align*}
D'autre part, $e^{t/10}-1$ peut \^{e}tre \'ecrire comme $\sum_{n=1}^{\infty}
\frac{t^{n}}{n!10^{n}}$. Apr\`es comparaison des coefficients, on obtient, \\
\[
	\sum_{k=1}^{n}\frac{1}{k!(n-k)!}\left( \frac{1}{10^{k}} -
	\frac{9^{k}}{10^{n+1}} \right)  \alpha_{n-k} = \frac{1}{n!}\frac{1}{10^{n}} 
\]
Cette \'equation peut-\^etre \'ecrit sous la forme:
\begin{equation}
	\sum_{k=1}^{n} \binom{n}{k}\left(10^{n-k+1}-9^{k}\right) \alpha_{n-k} = 10 
	\label{eq:alpha}
\end{equation}
% ou directement,
% \begin{equation}
% 	\alpha_{n} = 
% \end{equation}
De l'\'equation (\ref{eq:alpha}), on peut obtenir $\alpha_{0} = 10$,
$\alpha_{1} = 360/91$, etc. De la m\^{e}me mani\`ere, en utilisant $g(x)
= (e^{t/10} - 1)e^{t(1-x)}$, on peut obtenir,
\begin{equation}
	\sum_{k=1}^{n} \binom{n}{k}\left(10^{n-k+1}-10^{k}+1\right) \beta_{n-k} =
	10(11^{n}-10^{n}) 
	\label{eq:beta}
\end{equation}
De l'\'equation (\ref{eq:beta}), on peut obtenir $\beta_{0} = 10$, $\beta_{1}
= 550/91$, etc. Pour appliquer ces r\'esultats \`a l'\'equation (\ref{eq:sx}),
on a besoin d'approcher $r(x)$ par un polyn\^{o}me. Puisque on recherche $l(r)$
et on juste trouv\'e comment on peut \'ecrire $l(x^{n})$ pour $n$ arbitraire.
Mais le d\'eveloppement limit\'e de $r(x)$, converge tr\`es lentement dans
$[0,1]$. On doit trouver une meilleure solution.

\subsubsection{Une m\'ethode analytique}
% \addcontentsline{toc}{subsection}{Une M\'ethode Analytique}
On a trouv\'e une solution unique de l'\'equation (\ref{eq:sx}) dans $C[0,1]$,
mais il existe des solutions non born\'ees. Par exemple, une solution est
$-\frac{1}{x}$, et peut \^{e}tre v\'erifi\'ee par substitution. Cela signifie que
la fonction non born\'ee $s_{1}(x) = s(x) + \frac{1}{x}$ satisfait l'\'equation
$s_{1}-As_{1}=0$. Cependant, le fonction Digamma, $\psi(x) =
\frac{\Gamma'(x)}{\Gamma(x)}$ satisfait à l'équation suivante,
\cite[6.4.8]{handbook}, o\`u $n=0$ \& $m = 10$:
\begin{equation*}
	\psi(10x) = \ln(10) + \frac{1}{10}\sum_{k=0}^{9}\psi\left(x +
	\frac{k}{10}\right) \Rightarrow \psi(x) = \ln(10) +
	\frac{1}{10}\sum_{k=0}^{9}\psi\left( \frac{x+k}{10}\right).
\end{equation*}
Maintenant, on peut voir que la fonction $s_{2}(x)=s_{1}(x) +\psi(x) +\gamma$
satisfiant l'\'equation,
% s_2(x) - As_2(x) = \ln(10) + \frac{1}{10} \left[ \psi\frac{x+9}{10} +\gamma
% \right] = r_1(x)
\begin{align*}
		s_{2}(x) - As_{2}(x) &= \cancelto{0}{s_{1} - As_{1}}+ \psi - A\psi +
		\gamma - A\gamma \\
		&= \ln(10) + \frac{1}{10}\sum_{k=0}^{9}\psi\left( \frac{x+k}{10} \right)
		- \frac{1}{10}\sum_{k=0}^{8}\psi\left( \frac{x+k}{10} \right) + \gamma -
		\frac{9\gamma}{10}\\
		&= \ln(10) + \frac{1}{10} \left[ \psi\left( \frac{x+9}{10} \right) +
		\gamma \right] = r_{1}(x).
\end{align*}
Aussi, puisque $s_{1}(x) = s(x) + \frac{1}{x}$, on peut remplacer dans $s_{2}$
et en utilisant \cite[6.3.5]{handbook} on obtient,
\[
	s_{2}(x)= s(x) + \frac{1}{x} + \psi(x) + \gamma = s(x) + \psi(x+1) + \gamma.
\]
	Alors, $s_{2}$ est aussi en $C[0,1]$, et l'\'equation $s(0) = s_{2}(0) =
	l(r_{1})$ est vraie gr\^{a}ce \`a \cite[6.3.2]{handbook}. On peut aussi
	relier $r_{1}$ avec le fonction de zeta avec \cite[6.3.14]{handbook}. Donc,
\[
	r_1(x) = \ln(10) + \frac{1}{10}\left[ \psi \left( 1-\frac{1-x}{10} \right)
+\gamma \right] = \ln(10) - \frac{1}{10}\sum_{n=2}^\infty \zeta(n) \left(
\frac{1-x}{10} \right)^{n-1}.
\]
Mais, on recherche $s(0)$, ou de manière équivalente $l(r_{1})$. On a donc:
\begin{equation*}
	\begin{split}
		\sum_{m\in M}\frac{1}{m} &= l(r_{1})=l(\ln(10))-l(\sum_{n=2}^{\infty}
		\zeta(n) (1-x)^{n-1} 10^{-n}) \\
		&=\ln(10)l((1-x)^0)-\sum_{n=2}^{\infty}\zeta(n)10^{-n} l((1-x)^{n-1})\\
		&=\beta_{0}\ln(10) -\sum_{n=2}^{\infty}\zeta(n)10^{-n} \beta_{n-1}
	\end{split}
\end{equation*}
On peut aussi trouver l'erreur de troncature. Puisque $\beta_{n}$ et $\psi_{n}$
sont decroissantes, le rapport de deux termes est plus petit que $0.1$. Ce
pourquoi, l'erreur est plus petite que $0.1 + 0.001 + \ldots = 1/9$ fois ea
dernier terme inclus. Maintenant, on va proposer une nouvelle methode pour
calculer avec beaucoup plus de precision.

\subsubsection{Approximation de Chebyshev}
% \addcontentsline{toc}{subsection}{Approximation de Chebyshev}
\noindent Maintenant on va utiliser les polyn\^{o}mes de Chebyshev de premi\`ere
esp\`ece,
\[
	T_{n}(y) = \cos(n \arccos(y))
\]
On a donc,
\[
	|T_{n}(y)| \le 1 \quad \text{pour} \quad y\in [-1, 1] \Rightarrow
	|T_{n}(1-2x)| \le 1 \quad \text{pour} \quad x\in [0, 1].
\]
Le polyn\^{o}miel $T_{n+1}(3) - T_{n+1}(1-2x)$ vaut z\'ero quand $x=-1$, donc
est divisible par $1+x$. Alors on definit $Q_{n}(x)$ comme, $(T_{n+1}(3) -
T_{n+1}(1-2x))/(1+x)$. On a donc,
\[
	\frac{T_{n+1}(3)-1}{1+x} \le \frac{T_{n+1}(3) - T_{n+1}(1-2x)}{1+x} =
	Q_{n}(x) \le \frac{T_{n+1}(3)+1}{1+x}
\]
puisque $|T_{n+1}(1-2x)| \le 1$. Apr\`es, on divise $Q_{n}(x)$ par
$T_{n+1}(3)$ et on d\'efinit,
\[
	\frac{Q_{n}(x)}{T_{n+1}(3)} = \sum_{k=0}^{n} q_{n_{k}} x^{k}
\]
on a
\[
	\left( 1-\frac{1}{T_{n+1}(3)} \right)\frac{1}{1+x} \le
	\sum_{k=0}^{n}q_{n_{k}}x^{k} \le \left(
	1+\frac{1}{T_{n+1}(3)}\right)\frac{1}{1+x} \quad \text{pour} \quad x\in [0,1]
\]
et parce que
\[
	\frac{1}{m+x} = \frac{1}{m}\frac{1}{1+x/m} \quad \text{pour} \quad m=1,2,...,8
\]
En remplaçant $x$ par $x/m$ et en multipliant chaque côté des inégalités
par $1/m$, on obtient
\[
	\left( 1-\frac{1}{T_{n+1}(3)} \right)\frac{1}{m+x} \le
	\sum_{k=0}^{n}q_{n_{k}}\frac{1}{m^{k+1}}x^{k} \le \left(
	1+\frac{1}{T_{n+1}(3)}\right)\frac{1}{m+x} \quad \text{pour} \quad x\in
	[0,1]
\]
En sommant ces in\'egalit\'es pour $m=1, \ldots, 8$ et appliquer $l$, on a

\begin{equation}
	\left( 1-\frac{1}{T_{n+1}(3)} \right)\sum_{m\in M}\frac{1}{m} \le
	\sum_{k=0}^{n}q_{n_{k}}\alpha_{k}H_{k+1} \le \left(
	1+\frac{1}{T_{n+1}(3)}\right)\sum_{m\in M}\frac{1}{m} 
\end{equation}
o\`u $H_{k+1}$ est d\'efini par,
\[
	H_{k} = 1 + \frac{1}{2^{k}} + \ldots + \frac{1}{8^{k}}
\]
De plus, en utilisant \cite[1.49]{chebyshev}, on peut r\'e\'ecrire la
polyn\^{o}me da Chebyshev de premi\`ere esp\`ece de mani\`ere non r\'ecursive. 
\[
	T_n(x) = \frac{1}{2} \left[ (x+\sqrt{x^2-1})^n + (x-\sqrt{x^2-1})^n \right]
\]
Cette formule nous donne une estimation de degr\'e suffisant pour une
precision donn\'ee. On a,
\[
	T_{n}(3) = \frac{1}{2} ((3+2\sqrt{2})^{n} + (3-2\sqrt{2})^{n}) \approx
	\frac{1}{2}(3+2\sqrt{2})^{n} \qquad \text{pour les grands } n.
\]
Alors si on veut 50 chiffres de pr\'ecision, on doit calculer la polyn\^{o}me de
degr\'e 70, puisque $T_{70}(3) \approx 10^{53}$, alors,
\[
	(1 - 10^{-53})\sum_{m\in M}\frac{1}{m} \le \sum_{k=0}^{n}q_{n_{k}}
	\alpha_{k}H_{k+1} \le (1 + 10^{-53})\sum_{m\in M}\frac{1}{m}.
\]
Si on fait le calcul pour $n = 70$, et le terme d'erreur est $10^{-53}$. On a,\\
\fbox{\small\texttt{22.9206766192 6415034816 3657094375 9319149447 6243699848
1586404710 48521259}} \\
$\le s(0) \le$\\
\fbox{\small\texttt{22.9206766192 6415034816 3657094375 9319149447 6243699848
1545840772 76734915}}\\
Comme on peut voir, on a trouv\'e la valeur de somme pour une pr\'ecision de 52
chiffres.
