\documentclass{beamer}

% Setup appearance:
\usepackage{xcolor}
\usepackage{svg}
\usepackage{minted}
\usepackage{booktabs}
% Standard packages
\usepackage[french]{babel} % For French language support
\usepackage[utf8]{inputenc} % UTF-8 encoding for input
\usepackage[T1]{fontenc} % T1 encoding for output
\usepackage{lmodern} % Latin Modern font for better T1 encoding support
\usepackage{seqsplit}

% Setup TikZ
\usepackage{tikz}
\usepackage{tikz-cd}
\usepackage{adjustbox}
\usetikzlibrary{arrows}
\tikzstyle{block}=[draw opacity=0.7,line width=1.4cm]



\newcommand{\A}{\mathcal{A}}
\newcommand{\sump}{\sideset{}{'}\sum}
\newcommand{\norm}[1]{\left\lVert#1\right\rVert}

\definecolor{dark-charcoal}{RGB}{30,30,30}
\definecolor{steel-blue}{RGB}{70,130,180}
\definecolor{burgundy}{RGB}{100,0,25}
\definecolor{dark-slate}{RGB}{47,79,79}
\definecolor{light-ash}{RGB}{200,200,200}
\definecolor{text-black}{RGB}{20,20,20}
\definecolor{gray2}{RGB}{162,163,162}
\definecolor{LightGray}{gray}{0.9}

\usetheme[structurebold, palette=dark-charcoal]{Berkeley}
\usefonttheme[onlylarge]{structurebold}
\setbeamerfont*{frametitle}{size=\normalsize,series=\bfseries}
\setbeamertemplate{navigation symbols}{}
\setbeamertemplate{blocks}[rounded][shadow=true]

\setbeamercolor{structure}{fg=burgundy}
\setbeamercolor{section in head/foot}{bg=dark-slate, fg=light-ash}
\setbeamercolor{subsection in head/foot}{bg=dark-slate, fg=light-ash}
\setbeamercolor{frametitle}{bg=dark-charcoal, fg=white}
\setbeamercolor{title}{bg=steel-blue, fg=light-ash}
\setbeamercolor{author}{fg=text-black}
\setbeamercolor{institute}{fg=text-black}
\setbeamercolor{date}{fg=text-black}
\setbeamercolor{logo}{bg=gray2}



% Author, Title, etc.
\title 
{%
  Sommes Restreintes de Chiffres%
}


\author[Yusuf Emin Akpinar]
{
	{Yusuf Emin Akpinar\\[1ex]
	{\small \textbf{Superviseur}: Arnaud Bodin}}
}



\date
{Projet de Fin d'Etudes, 2025}
\logo{
\includesvg[width=1cm]{logo.svg}
}

% Définition des nouveaux environnements de théorèmes
\usepackage{amsthm} % nécessaire pour \newtheorem

% Define theorem styles with French names
\newtheorem{thm}{Théorème}
\newtheorem{lem}{Lemme}
\newtheorem{prop}{Proposition}
\newtheorem{cor}{Corollaire}
\newtheorem{defn}{Définition}
\newtheorem{exm}{Exemple}
\newtheorem{rmq}{Remarque}

% The main document

\begin{document}

\begin{frame}{Titre}
  \titlepage
\end{frame}

\begin{frame}{Contents}
  \tableofcontents
\end{frame}


\section{Introduction}

\subsection{Kempner}

\begin{frame}{Constraintes}
	Quelle contrainte peut-on imposer à une série harmonique pour la rendre
	convergente ?
	% Le premier \`a avoir apport\'e une r\'eponse \`a cette question est Kempner.
	% En 1914, il publie un article intitul\'e \textit{A Curious Convergent
	% Series}, dans lequel il d\'emontre que la s\'erie harmonique, priv\'ee des
	% termes dont le d\'enominateur contient le chiffre 9, devient convergente.
	\begin{block}{Th\'eor\`eme (Kempner, \textit{A Curious Convergent Series},
		1914)}
	La somme des inverses des entiers ne contenant pas le chiffre 9 converge :
	\[ K = \sum_{\substack{n\geq1 \\ \text{9 non dans } n}} \frac{1}{n} < 80. \]
	\end{block}
\end{frame}

\begin{frame}{Les travaux de Kempner}
	Partition de K selon le nombre de chiffres dans le dénominateur.

	\vspace{0.5em}

	\begin{itemize}
	  \item \( a_1 = 1 + \frac{1}{2} + \cdots + \frac{1}{8} < 8 \cdot 1 \)
	  \item \( a_2 = \frac{1}{10} + \frac{1}{11} + \cdots + \frac{1}{18} +
		  \frac{1}{20} + \cdots \frac{1}{28} + \frac{1}{30} + \cdots+
		  \frac{1}{88} < 8 \cdot 9/10 \)
	  \item \dots
	  \item $ a_{n} < 8 \cdot \left(9/10\right)^{n-1}$
	  \item $K \le 8 \sum_{n} (9/10)^{n} = 80$
	\end{itemize}

	\vspace{0.5em}

	% Avec une somme geométrique simple, nous avons $K\le 8\cdot 10 = 80$. Alors
	% on sait que la somme de Kempner converge et plus est petite que 80.
\end{frame}


% \begin{frame}{Majorations des sous-séries \( a_n \)}
% 	On majore chaque sous-série \( a_n \) en utilisant une borne supérieure sur
% 	les termes :
% 	\vspace{0.5em}
% 	\begin{equation*}
% 		\begin{split}
% 			\forall s \in a_1, \; s < 1 \quad &\Rightarrow \quad \sum a_1 = S_1
% 			< 1 \cdot 8\\
% 			\forall s \in a_2, \; s < \frac{1}{10^{1}} \quad &\Rightarrow \quad
% 			\sum a_2 = S_2 < \frac{1}{10} \cdot 8 \cdot 9 = 8 \cdot
% 			\frac{9}{10}\\
% 			& \text{  } \vdots\\
% 			\forall s \in a_{n}, \; s< \frac{1}{10^{n-1}} \quad &\Rightarrow
% 			\quad \sum a_{n} = S_{n} < 8 \cdot \left(\frac{9}{10}\right)^{n-1}
% 		\end{split}
% 	\end{equation*}
% 	\vspace{0.5em}
%
% 	Avec une somme geométrique simple, nous avons $K\le 8\cdot 10 = 80$. Alors
% 	on sait que la somme de Kempner converge et plus est petite que 80.
% \end{frame}

\subsection{Irwin}

\begin{frame}{M\'ethode d'Irwin}
	% Deux ans plus tard, Irwin a trouv\'e une bien meilleure approximation de \(
	% K \). Il compare les sous-s\'eries \( a_n \) introduites par Kempner.
	\( a_2 = \frac{1}{10} + \frac{1}{11} + \cdots + \frac{1}{18} + \frac{1}{20}
	+ \cdots+ \frac{1}{88}	\)\\

	Il compare les neuf premiers termes de $a_{3}$.

	\begin{itemize}
		\item $1/100 + 1/101 + \dots + 1/108 \le 9 \cdot 1/100=9/10 \cdot 1/10$ 
		\item $1/110 + 1/111 + \dots + 1/118 \le 9 \cdot 1/110=9/10 \cdot 1/11$
	\end{itemize}
	Donc nous avons
	\[
		a_{3} \le \frac{9}{10}a_{2}.
	\]
	Et plus généralement,
	\[
		a_{n} \le \left( \frac{9}{10} \right)^{n-2} a_{2}.
	\]
\end{frame}

\begin{frame}{M\'ethode d'Irwin}
	\[
		K \le a_{1} + \left[ 1+\frac{9}{10} +\left(\frac{9}{10}\right)^{2} +
		\dots\right] a_{2} = a_{1} + 10 a_{2}.
	\]
	% Irwin a trouv\'e que, $a_{1} \le 2,72$ et $10a_{2} \le 20,58$, donc le
	% somme, K est plus petite que 23,3.\\
	\begin{itemize}
		\item $a_{1} \le 2,72$
		\item $10a_{2} \le 20,58$
		\item $K < 23,3$
		\item $ 22,4 < K < 23,3$
	\end{itemize}


	% Pour la borne inf\'erieure, il applique la m\^eme m\'ethode mais cette fois,
	% la somme $a_{n}$ est plus grande que $(9/10)^{n-2}a_{2}{'}$ o\`u,

	% \begin{equation*}
	% 	a_{2}{'} = \frac{1}{11} + \frac{1}{12} + \cdots + \frac{1}{19} +
	% 	\frac{1}{21} + \frac{1}{22} + \cdots + \frac{1}{29} + \cdots +
	% 	\frac{1}{81} + \cdots + \frac{1}{89}
	% \end{equation*}
	% et, donc la somme est plus grande que,
	% \[
	% 	a_{1} + a_{2} + 9\cdot a_{2}{'}
	% \]
	% Pour la borne inf\'erieure, qu'il a calcul\'e comme 22,4.\\
	% Avec ce m\'ethode, j'ai obtenu six chiffres apr\`es la virgule en
	% commen\c{c}ant \`a partir de \( a_7 \) au lieu de \( a_2 \).
\end{frame}

\section{Théorème du binôme}

\subsection{Baillie}

\begin{frame}{Estimation de Baillie (1979)}
	% En 1979, Robert Baillie utilise la formule binomiale pour dériver un formule
	% récursive pour le somme de Kempner. Il a une idée astucieux de travailler
	% avec la somme des puissances des inverses et définit :
	\begin{itemize}
		\item Formule binomiale
		\item Puissances des inverses
		\item Plus g\'en\'eral, pas seulement 9.
	\end{itemize}

	% \[
	% 	S_{i+1}=\bigcup_{s\in S_{i}}\{10s,10s+1,\ldots,10s+9\}\backslash\{10s+m\},
	% \]
	% et
	\[
		s(i+1,j) = \sum_{x\in S_{i+1}}\frac{1}{x^{j}} = \sum_{x\in S_{i}}
		\sum_{\substack{k=0\\ k\ne m}}^{9} (10x+k)^{-j}.
	\]
	% Apr\`es on applique le th\'eor\`eme binomial n\'egatif et on obtient
	% \[
	% 	(10x+k)^{-j} = (10x)^{-j}\sum_{n=0}^{\infty} (-1)^{n} \binom{j+n-1}{n}
	% 	(k/10x)^{n}.
	% \]
\end{frame}

\begin{frame}{Estimation de Baillie}
	Apr\`es les calculs, on obtient,
	\[
		s(i+1, j) = \sum_{n=0}^{\infty}a(j,n)s(i, j+n),
	\]
	o\`u
	\begin{align*}
		&c(j, n) = (-1)^{j} \binom{j+n-1}{n}\\
		&b_{n}=1^{n}+\ldots + 9^{n} - m^{n} \hspace{20pt} (n\ge0), b_{0} = 9\\
		&a(j, n) = b_{n}c(j, n)/10^{j+n}.
	\end{align*}
	% \begin{exm}
	% 	\begin{equation*}
	% 		\begin{split}
	% 			s(2, 1) &= \sum_{n=0}^{\infty} a(1, n) s(1, 1+n) \\
	% 					&= a(1, 1)s(1, 2) + a(1, 2)s(1, 3) + \ldots
	% 		\end{split}
	% 	\end{equation*}
	% \end{exm}
\end{frame}

\begin{frame}{Estimation de Baillie}
	On a donc,
	\begin{itemize}
		\item[\textbullet] Ainsi, pour $i\le4$, on calcule $s(i, j)$ explicitement.
		\item[\textbullet] Pour $5\le i\le30$, on utilise la formule de r\'ecurrence.
		\item[\textbullet] Pour $i\ge31$, on va utiliser l'estimation,
	\end{itemize}
	\begin{gather*}
		\sum_{i=31}^{\infty} s(i,1) \approx 9 \cdot s(30, 1) 
	\end{gather*}
	\[
		K = \sum_{i=1}^{\infty} s(i, 1) = 22,92067661926415034816\ldots
	\]
\end{frame}
	
\subsection{Schmelzer \& Baillie}

% \begin{frame}{Schmelzer \& Baillie (2008)}
% 	\begin{itemize}
% 	  \item Extension : motifs très variés (ex. "42", "314").
% 	  \item Comportement asymptotique de la somme :
% 	  \[ \lim_{n \to \infty} \frac{\Psi_{X_n}}{10^n} = \frac{10^p}{10^p - 1} \ln(10) \]
% 	\end{itemize}
% \end{frame}

\begin{frame}{D\'efinitions de Schmelzer et Baillie}
	\begin{itemize}
		\item Motifs plus g\'en\'eral,
		\item Comportement asymptotique
		\item $X$ une cha\^{i}ne de $n$ chiffres
		% \item $S_{i}^{j}$: Les elements de $S_{i}$ qui les derniers $j-1$
		% 	chiffres sont \'egaux avec premiers $j-1$ chiffres de $X$.
	\end{itemize} 

	Matrice de $X=9$, cas de Kempner,
	\begin{equation*}
			T = 
		\left[
			\begin{array}{c|cccccccccc}
						& 0 & 1 & 2 & 3 & 4 & 5 & 6 & 7 & 8 & 9 \\
				\hline 1& 1 & 1 & 1 & 1 & 1 & 1 & 1 & 1 & 1 & 0 \\
			\end{array}
		\right]
	\end{equation*}

	Matrice de $X=42$,
	\begin{equation*}
			T = 
		\left[
			\begin{array}{c|cccccccccc}
						& 0 & 1 & 2 & 3 & 4 & 5 & 6 & 7 & 8 & 9 \\
				\hline 1& 1 & 1 & 1 & 1 & 2 & 1 & 1 & 1 & 1 & 1 \\
				2		& 1 & 1 & 0 & 1 & 2 & 1 & 1 & 1 & 1 & 1 \\
			\end{array}
		\right]
	\end{equation*}

	% AJOUTER VALUER DE K POUR X=42.
	$K = \seqsplit{228.44630415923081325414808612625058957816292753983036118591
	34600004528607686502143070480461174144321741} \ldots$

	% Soit, X une cha\^{i}ne de $n$ chiffres. On d\'efinit $S_{i}^{j}$ comme
	% l'ensemble de tous les membres de $S_{i}$ dont les derniers $j-1$ chiffres
	% sont exactement \'egaux avec les premiers $j-1$ chiffres de $X$, mais dont
	% les derniers $j$ chiffres sont diff\'erents des premiers $j$ chiffres de
	% $X$.\\
\end{frame}

% \begin{frame}[fragile]{Matrice de r\'ecurrence}
% 	% \begin{equation*}
% 	% \adjustbox{scale=0.7}{%
% 	% 	\begin{tikzcd}
% 	% 		{S^1} && {S^2}
% 	% 		\arrow[from=1-1, to=1-1, loop, in=145, out=215, distance=10mm]
% 	% 		\arrow[latex-latex, from=1-1, to=1-3]
% 	% 		\arrow[latex-latex, no head, from=1-3, to=1-3, loop, in=35, out=325,
% 	% 		distance=10mm]
% 	% 	\end{tikzcd}}
% 	% \end{equation*}
%
% 	\vspace{3em}
%
%
% 	% Et nous définissons un tenseur $f$ lié à notre matrice $T$.
% 	% \[
% 	% 	f_{jlm} = 
% 	% 	\begin{cases}
% 	% 		1 & \text{si }  T(l,m) = j \\
% 	% 		0 & \text{sinon}.
% 	% 	\end{cases}
% 	% \]
% 	%
% 	% \[ 
% 	% 	A_{n} = \begin{pmatrix}
% 	% 		f_{11m} & \ldots & f_{1nm} \\
% 	% 		\vdots  &        & \vdots \\
% 	% 		f_{n1m} & \ldots & f_{nnm} 
% 	% 	\end{pmatrix}.
% 	% \]
% \end{frame}

% \begin{frame}{Troncature}
% 	Enfin, pour un grand $M$ il a trouve la troncature comme suivant:
% 	\begin{equation*}
% 		K \approx \sum_{j=1}^{n}\sum_{i=1}^{M}K_{i}^{j} + \norm{B
% 		\begin{pmatrix}
% 			K_{M}^1 \\
% 			\vdots \\
% 			K_{M}^n 
% 		\end{pmatrix}
% 		}_{1},
% 	\end{equation*}
% 	o\`u
% 	\begin{equation*}
% 		\begin{split}
% 			&K_{i}^{j} = \sum_{s\in S_{i}^{j}} \frac{1}{s}\\
% 			&B = \sum_{k=1}^{\infty}A_{n}^{k}=(I_{n}- A_{n})^{-1}-I_{n}.
% 		\end{split}
% 	\end{equation*}
% \end{frame}

% \begin{frame}[fragile]{Comportement}
% 	\begin{itemize}
% 		\item Longueur du motif
% 		\item Périodicité
% 	\end{itemize}
%
% 	\vspace{3em}
%
% 	\begin{table}[ht!]
% 		\mbox{}\hfill
% 		\begin{minipage}{0.45\textwidth}
% 			\vfill{}
% 			\resizebox{\columnwidth}{!}{%
% 				\begin{tabular}{@{}r@{\hspace{5mm}}l@{\hspace{5mm}}l@{}}
% 					\toprule
% 					$n$ & Cha\^ine $X_n$ & $K/10^n$ \\ \midrule
% 					20  & 21794968139645396791 & 2,30258 50929 94045 68397 52162\\
% 					20  & 31850115459210380210 & 2,30258 50929 94045 68399 08824\\
% 					20  & 67914499976105176602 & 2,30258 50929 94045 68401 09579\\
% 					20  & 98297963712691768117 & 2,30258 50929 94045 68401 77079\\ \bottomrule
% 				\end{tabular}%
% 			}
% 		\end{minipage}%
% 		\hfill{}
% 		\begin{minipage}{0.45\textwidth}
% 			\vfill{}\ \\
% 			\resizebox{\columnwidth}{!}{%
% 				\begin{tabular}{@{}r@{\hspace{5mm}}l@{\hspace{5mm}}l@{}}
% 					\toprule
% 					$n$ & Cha\^ine $X_n$ & $K/10^n$ \\ \midrule
% 					20  & 00000000000000000000 & 2,55842 78811 04495 20443 88506 \\
% 					20  & 99999999999999999999 & 2,55842 78811 04495 20443 88506 \\
% 					20  & 42424242424242424242 & 2,32584 35282 76813 82219 89695 \\
% 					20  & 09090909090909090909 & 2,32584 35282 76813 82221 85405 \\ \bottomrule
% 				\end{tabular}%
% 			}
% 		\end{minipage}\hfill
% 		\mbox{}
% 	\end{table}
% 	% Schmelzer et Baillie ont observé que le résultat ne dépend pas du motif,
% 	% mais seulement de la longueur et du type du motif.
% \end{frame}

\section{Theorie de la mesure}
\subsection{Jean-François Burnol}

\begin{frame}{D\'efinitions: Burnol}
	\begin{itemize}
		\item $\A$: Entiers non n\'egatifs admissibles
		\item $l(n)$: Nombre de chiffres de l'entier n
		\item $\sum_{n}^{'}$: $\sum_{n \in \A}$
		\item $\mu = \sum_{l\ge0} 10^{-l}\sum^{'} \delta_{n/10^{l}}$

		\item $\int_{I} f(x)d\mu(x)= \sum_{l\ge0}10^{-l}\sum_{n/10^{l} \in I}^{'}
			f(n/10^{l})$
	\end{itemize}

	\vspace{0.5em}

	\[
		K = \int_{[1/10, 1)} \frac{d\mu(x)}{x}.
	\]

\end{frame}

\begin{frame}{Th\'eor\`eme Principal}
	\begin{thm}
		\[
			K = \sump_{0<n<10^{l-1}}\frac{1}{n} +
			10\sump_{l(n)=l}\frac{1}{n} +
			\sum_{m=1}^{\infty}(-1)^{m}u_{m}\beta_{l, m+1}.
		\]
	\end{thm}
	\[
		u_{m} = \int_{[0,1)}x^{m}d\mu(x) = \sum_{l\ge0}10^{-l}\sump_{n/10^{l}
		\in [0,1)} (n/10^{l})^{m} \qquad (m\ge0).
	\]
	\vspace{0.5em}
	\[
		\beta_{l, m} = \sump_{l(n)=l}n^{-m},
	\]
\end{frame}

\begin{frame}{Formula}
	% \begin{equation*}
	% 	\begin{split}
	% 		K &= \sump_{0<n<10^{l-1}} \frac{1}{n} + \sump_{n\ge
	% 		10^{l-1}}\frac{1}{n}\\
	% 		  &= \sump_{0<n<10^{l-1}} \frac{1}{n} + \sump_{n\in
	% 		  [10^{l-1},10^{l})} \sump_{ld_{l}(m)=n}\frac{1}{m}.
	% 	\end{split}
	% \end{equation*}
	\begin{lem}
	Pour tout $n\in\A$ non nul de longueur $l(n)$,
	\[
		\int_{[0, 1)}\frac{1}{n+x} d\mu(x) = \sump_{ld_{l(n)}(m)=n}\frac{1}{m}.
	\]
	\end{lem}
	\[
		K = \sump_{0<n<10^{l-1}} \frac{1}{n} + \sump_{l(n)=l} \int_{[0,1)}
		\frac{1}{n+x}d\mu(x)
	\]
	La série de Taylor de $(n+x)^{-1}$ nous donne,
	\[
		\int_{[0,1)} \frac{1}{n+x} d\mu(x) = \int_{[0, 1)} \sum_{k=0}^{\infty}
		(-1)^{k} \frac{x^{k-1}}{n^{k}} d\mu(x) = \frac{u_{0}}{n} -
		\frac{u_{1}}{n^{2}} + \frac{u_{2}}{n^{3}}+\ldots
	\]
	% By using this lemma, we can deduce that
	% En utilisant ce lemme, nous pouvons déduire que
\end{frame}

% \begin{frame}{Formula}
% 	\[
% 		K = \sump_{0<n<10^{l-1}} \frac{1}{n} + \sump_{l(n)=l} \int_{[0,1)}
% 		\frac{1}{n+x}d\mu(x)
% 	\]
% 	La série de Taylor de $(n+x)^{-1}$ nous donne,
% 	\[
% 		\int_{[0,1)} \frac{1}{n+x} d\mu(x) = \int_{[0, 1)} \sum_{k=0}^{\infty}
% 		(-1)^{k} \frac{x^{k-1}}{n^{k}} d\mu(x) = \frac{u_{0}}{n} -
% 		\frac{u_{1}}{n^{2}} + \frac{u_{2}}{n^{3}}+\ldots
% 	\]
% 	% So if we can replace it, we got:
% 	Donc, si nous le réécrivons, nous avons :
% 	\begin{equation*}
% 		\begin{split}
% 			K &= \sump_{0<n<10^{l-1}} \frac{1}{n} + \sum_{m=0}^{\infty}
% 			(-1)^{m}u_{m} \sump_{l(n)=l} \frac{1}{n^{m+1}} \\
% 			  &= \sump_{0<n<10^{l-1}} \frac{1}{n} + \sum_{m=0}^{\infty}
% 			  (-1)^{m}u_{m} \beta_{l, m+1}.
% 		\end{split}
% 	\end{equation*}
% \end{frame}

\begin{frame}{Observations}
	% Donc, pour pouvoir calculer la valeur de $K$, nous devons calculer $u_{m}$,
	% $\beta_{l, m+1}$ et une somme finie de $\frac{1}{n}$, pour un $l$ fixé. En
	% fait, $\beta$ est aussi une somme finie, donc la vraie difficulté réside
	% dans le calcul de $u_{m}$. Pour pouvoir calculer $K$, nous avons besoin
	% d'une formule efficace pour $u_{m}$.
	Pour calculer $K = \sum_{0<n<10^{l-1}}^{'}\frac{1}{n} + 10 \sum_{l(n)=l}^{'}
	\frac{1}{n} + \sum_{m=1}^{\infty}(-1)^{m}u_{m}\beta_{l, m+1}$,
	\begin{itemize}
		\item Somme finie
		\item $\beta_{l, m+1}$
		\item Calcul de $\alert{u_{m}}$ 
	\end{itemize}
	% So to be able to compute the value of $K$, we need to compute $u_{m}$,
	% $beta_{l, m+1}$ and finite sum of $\frac{1}{n}$, for fixed $l$. Actually
	% $\beta$ is also a finite sum, so the real difficulty lies in the computation
	% of $u_{m}$. To be able to compute $K$, we need an efficient formula for
	% $u_{m}$.
\end{frame}

\begin{frame}{Estimation des moments}
	\begin{lem}
		Soit f est une function born\'e sur $[0, 10)$. Alors
		\[
			\int_{[0, 1)} f(10x) d\mu(x) = f(0) + \int_{[0, 1)} \frac{1}{10}
			\sum_{a=0}^{8} f(a+x) d\mu(x).
		\]
	\end{lem}
	% En utilisant de ce lemme, on peut trouver une formule r\'ecursif pour les
	% moments $u_{m}$. Si on prendre $f(x) = x^{m}$ on a,
	Prendre $f(x) = x^{m}$.
	% \[
	% 	10^{m+1}u_{m} = \int_{[0, 1)} \sum_{a\in A}(a+x)^{m}d\mu(x) =
	% 	9u_{m} + \sum_{j=1}^{m} \binom{m}{j}u_{m-j}\gamma_{j}.
	% \]
	% On a donc,
	\[
		(10^{m+1}-9)u_{m} = \sum_{j=1}^{m}\binom{m}{j}\gamma_{j}u_{m-j}, \qquad
		(m \ge 1).
	\]
\end{frame}

\section{Conclusion}

\subsection{Computation et implementation}

\begin{frame}[fragile]{Obstacle}
	% Premier obstacle que j’ai eu : nous avons besoin de tous les coefficients
	% binomiaux, donc les calculer à l’aide des fonctions intégrées, qui utilisent
	% la fonction gamma pour les obtenir directement, est très lent comparé à la
	% méthode récursive.
	\begin{itemize}
	\item Coefficients binomiaux
	\begin{minted}[bgcolor=LightGray]{python}
Sagemath prendre 29.60158967300049 secondes.
Triangle prendre 0.16335931699904904 secondes.
	\end{minted}
	\item Le calcul de $\beta$ est facile et parall\'elisable
	\item Comme \alert{$u_{m}$} est definit de façon r\'ecursive ce n'est pas
		parall\'elisable.
	\item Acc\'el\'eration par pr\'ecision dynamique
	\end{itemize}

% 	\begin{minted}{python}
% def builtin(limit=1000):
%     for i in range(1, limit):
%         for j in range(i):
%             x = scipy.special.comb(i, j, exact=True)
% def triangle(limit=1000):
%     combi = [[1],[1, 1]]
%     for j in range(2, limit):
%         L = [1]
%         for i in range(j):
%             cur_comb =  combi[j-1][i] + combi[j-1][i-1] 
%             L.append(cur_comb)
%         combi.append(L)
% 	\end{minted}
\end{frame}

% \begin{frame}[fragile]{Comparaison of methods}

	% \begin{itemize}
	% 	\item On doit utiliser la methode r\'ecursive
	% 	\item Nous pouvons utiliser les propriétés symétriques des coefficients
	% 		binomiaux pour n’en calculer que la moitié.
	% 	\item Nous pouvons traiter les binomiaux par lots, donc d’abord calculer
	% 		les $n$ premières lignes, puis écraser ces lignes avec les $n$
	% 		lignes suivantes.
	% 	% \item We can use the symmetric properties of binomial coefficients to
	% 	% 	calculate only half of it.
	% 	% \item We can batch process the binomials, so first calculate first $n$
	% 	% 	lines, then override these line with second $n$ lines.
	% \end{itemize}
% \end{frame}

% \begin{frame}[fragile]{Calcul de $\beta$}
% 	\small
% 	\begin{minted}{c}
% #pragma omp parallel 
% {
%   #pragma omp for
%   for (int m = 0; m < LIMIT; m++) {
%     mpfr_init2(betas[m], precisionBits);
%     mpfr_set_ui(betas[m], 0, MPFR_RNDN);
%     for (ulongi n = start; n < end; n++) {
%       if (isAdmissible(n)) {
%         mpz_ui_pow_ui(power, n, m);
%         mpfr_div_z(term, one, power, MPFR_RNDN);
%         int comp = mpfr_cmp(term, betaLimit);
%         if (comp < 0) {
%           mpfr_add(betas[m], betas[m], term, MPFR_RNDN);
%           break;
%         }
%         mpfr_add(betas[m], betas[m], term, MPFR_RNDN);
% }}}} 
% 	\end{minted}
% \end{frame}

% \begin{frame}{Calcul de $u_{m}$}
% 	Le calcul de $\beta$ est à la fois facile et rapide grâce à sa nature
% 	facilement parallélisable. Cependant, on ne peut pas en dire autant pour
% 	$u_{m}$. Sa nature récursive ne nous permet pas de paralléliser le
% 	processus. Nous ne pouvons utiliser le multithreading que partiellement dans
% 	le calcul de $u_{m}$.
% \end{frame}

% \begin{frame}{Pr\'ecision dynamique}
% 	Au fil des calculs les derni\'ers $u_{m}$ ne contribuent qu'aux derniers
% 	chiffres d\'ecimaux. Donc, nous n'avons pas besoin de calculer tous les
% 	$u_{m}$ \`a la m\^{e}me pr\'ecision en bits. Nous pouvons r\'eduire
% 	dynamiquement la pr\'ecision pour calculer \`a la fois les $u_{m}$ et les
% 	$\beta_{m}$ pour acc\'el\'erer le calcul.
% \end{frame}


% \begin{frame}[fragile]{Calcul de $u_{m}$}
% 	\tiny
% 	\begin{minted}{c}
% void um(mpfr_t u[LIMIT], mpz_t gammas[LIMIT], mpz_t *combi[binomPerLine], int decr) {
%   int precisionBits = BITS+GUARD;
%   mpz_t den; mpz_init(den);
%   mpfr_t denf; mpfr_init2(denf, precisionBits);
%   mpfr_init2(u[0], precisionBits); mpfr_set_ui(u[0], 10, MPFR_RNDN);
%   for (int m = 1; m<LIMIT+1; m++) {
%     if (m % decr == 0) {
%       int newBits = BITS + GUARD - (int)((double)m/decr) * STEP;
%       precisionBits = (newBits > 0) ? newBits : precisionBits;
%     }
%     mpfr_init2(u[m], precisionBits); mpfr_set_zero(u[m], 0);
%     mpfr_t sumGlobal; mpfr_init2(sumGlobal, precisionBits);
%     mpfr_set_ui(sumGlobal, 0, MPFR_RNDN);
%     #pragma omp parallel
%     {
%       mpz_t binomial; mpz_init(binomial);
%       mpfr_t sumf; mpfr_init2(sumf, precisionBits);
%       mpfr_t temp; mpfr_init2(temp, precisionBits);
%       mpfr_set_ui(sumf, 0, MPFR_RNDN);
% 	\end{minted}
% \end{frame}
%
% \begin{frame}[fragile]{Calcul de $u_{m}$}
% 	\tiny
% 	\begin{minted}{c}
%       #pragma omp for nowait
%       for (int j = 1; j < m+1; j++) {
%         #pragma omp critical
%         {
%           get_binom(binomial, combi, m, j);
%         }
%         mpz_mul(binomial, binomial, gammas[j]);
%         mpfr_mul_z(temp, u[m-j], binomial, MPFR_RNDN);
%
%         mpfr_add(sumf, sumf, temp, MPFR_RNDN);
%       }
%       #pragma omp critical
%       {
%         mpfr_add(sumGlobal, sumGlobal, sumf, MPFR_RNDN);
%       }
%       mpz_clear(binomial);
%       mpfr_clears(sumf, temp, NULL);
%     }
%     mpz_ui_pow_ui(den, 10, m+1);
%     mpz_sub_ui(den, den, 9);
%     mpfr_set_z(denf, den, MPFR_RNDN);
%     mpfr_div(u[m], sumGlobal, denf, MPFR_RNDN);
%     mpfr_clear(sumGlobal);
%   }
%   mpz_clear(den);
%   mpfr_clear(denf);
% }
% 	\end{minted}
% \end{frame}

\begin{frame}{Resultat (pr\`emi\'ere 2000 termes)}
\noindent\tiny{\texttt{22.\\
9206766192 6415034816 3657094375 9319149447 6243699848 1568541998 3565721563
3818991112 9445626037 4482018989 9096412533 2346922160 4711904783 1029750614
6968857121 0180678649 3339402886 9627795786 8596119863 7905620169 3218804088
0170136179 0211062866 1173509921 1021080576 7037858147 1208344258 7658322726
5762010383 1470760370 3081599962 3544735896 5269056768 8849708196 0327431233
1458892799 7290413878 4952498149 4420459215 2773507367 0721852000 4083026308
9161691211 2386263685 9589823575 1717059249 8667879488 4732108924 8065916234
0101523560 0050654804 3749678309 0130313355 6109695301 4813317749 5576252380
5629716085 0098435454 7601825342 2157510734 4839216578 2984461954 2391601061
1783538353 9414385364 5608545221 8993239443 6643879041 5885760914 4227813991
9999224205 5353569500 6903416817 5189094448 0911928277 8344699965 1712608600
6663606677 8802880840 6885936480 2875179090 9188136795 1277973480 0336594138
0076337136 2027592352 3021897838 8060696159 3219106619 2832138116 9578671501
2908593756 7695180108 1088185294 6961772722 2369263351 0303284693 1322633320
4662982671 9621921949 7595341302 9846707264 1494803176 1513294759 2059710895
2767229950 6135926501 6527793966 5504378141 9771218981 1411533178 4288289120
0861629055 7888948801 9672973398 2154879558 6539443231 1715997509 0536480771
1452508644 0021477483 7830566983 2947534643 1574887600 2553447060 5016720081
3537681505 1011744737 0525476106 1212995726 7448260674 6619359970 9646499349
6193687192 2718731843 2066929586 4568048819 3966538192 5642051018 4836565613
7014651202 9952508182 0546550025 3512640541 2417351362 7643280141 1344172644
2573721215 8630804360 6232625102 1258024747 6381362063 5344177907 1562727116
1327878990 2233978650 7077973087 1328361981 8042590675 8456493642 1334639694
6418242998 8545122769 8638525759 1169510862 8538843908 9619061464 8926595578
2873997599 0384755992 0768769694 9762798020 9373018576 2618773856 2079135497
8209812211 0653711848 5627606686 3650232959 5606375781 3015885853 8424462891
4371644644 2718250845 0081189605 3012040621 9669846038 5157169114 2506412741
9331936610 8787735837 2424020359 9203103052 5230441065 4214782842 1259579767
3879911778 3670920609 0520816896 7959515743 
}}
\end{frame}

\end{document}


